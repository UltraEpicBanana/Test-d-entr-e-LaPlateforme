\documentclass[12pt,a4paper]{article}
\usepackage[utf8]{inputenc}
\usepackage[T1]{fontenc}
\usepackage[french]{babel}
\usepackage{geometry}
\geometry{margin=2.5cm}
\usepackage{hyperref}
\usepackage{listings}
\usepackage{xcolor}
\usepackage{fancyhdr}
\usepackage{titlesec}
\usepackage{array}
\usepackage{booktabs}

\titleformat{\section}{\normalfont\Large\bfseries}{Partie \thesection}{1em}{}
\titleformat{\subsubsection}{\normalfont\large\bfseries}{\thesubsubsection}{1em}{}

\definecolor{lightgray}{gray}{0.95}
\lstset{
  backgroundcolor=\color{lightgray},
  basicstyle=\ttfamily\small,
  frame=single,
  breaklines=true,
  postbreak=\mbox{\textcolor{red}{$\hookrightarrow$}\space}
}

\pagestyle{fancy}
\fancyhf{}
\rhead{Projet S\'ecurit\'e Serveur Linux}
\lhead{S\'ecurit\'e et supervision}
\cfoot{\thepage}

\title{\textbf{Projet Firewall : Test d'entrée à LaPlateforme en deuxième année}}
\author{Teva Clairefond}
\date{Juillet 2025}

\begin{document}

\maketitle
\tableofcontents
\newpage






\section{Questions}

\subsection{Système}

\subsubsection{Question 1 : Visualiser les processus SSH}

Commande pour afficher les processus SSH actifs :
\begin{lstlisting}
ps -ef | grep ssh
\end{lstlisting}

Pour une vue interactive :
\begin{lstlisting}
sudo apt install htop
htop
\end{lstlisting}

\subsubsection{Question 2 : Obtenir l'adresse IP de la machine}

\begin{lstlisting}
ip addr
\end{lstlisting}

\subsubsection{Question 3 : Hyperviseur de type 1 vs type 2}

\begin{itemize}
    \item \textbf{Type 1 (bare metal)} : Exécute directement sur le matériel. Utilisé en production pour virtualiser des serveurs dans les datacenters.
    \item \textbf{Type 2 (hosted)} : Fonctionne au-dessus d’un OS. Utilisé pour le test et le développement (ex : VirtualBox, VMware Workstation).
\end{itemize}

\vspace{0.5cm}

\subsection{Réseau}

\subsubsection{Question 1 : Protocoles, classification OSI et sécurité}

\subparagraph*{Classification des protocoles}

\begin{tabular}{>{\bfseries}l l}
\toprule
Couche OSI & Protocoles \\
\midrule
Application (7) & HTTP, HTTPS, FTP, SFTP, DNS, SMTP, IMAP, POP3, SSH \\
Transport (4) & TCP, UDP \\
Réseau (3) & IP (IPv4, IPv6) \\
Liaison de données (2) & Ethernet \\
\bottomrule
\end{tabular}

\vspace{0.5cm}

\subparagraph*{Protocoles sensibles au niveau sécurité}

\begin{itemize}
    \item \textbf{HTTPS / SSH} : Chiffrés, sécurisés.
    \item \textbf{DNS} : Vulnérable au spoofing (empoisonnement).
    \item \textbf{FTP} : Transfert non chiffré → écoute possible.
    \item \textbf{TCP} : Vulnérable aux attaques SYN flood, spoofing.
\end{itemize}

\subsubsection{Question 2 : Couches OSI associées}

\begin{tabular}{>{\bfseries}l l}
\toprule
Élément & Couche OSI \\
\midrule
Switch & Couche 2 – Liaison de données \\
Routeur & Couche 3 – Réseau \\
TCP & Couche 4 – Transport \\
HTTP & Couche 7 – Application \\
\bottomrule
\end{tabular}

\subsubsection{Question 3 : Enregistrements DNS}

\begin{itemize}
    \item \textbf{A record} : Lien nom de domaine $\rightarrow$ IPv4.
    \item \textbf{AAAA record} : Lien nom de domaine $\rightarrow$ IPv6.
\end{itemize}

\subsubsection{Question 4 : SYN et ACK}

\begin{itemize}
    \item \textbf{SYN (Synchronize)} : Demande de connexion.
    \item \textbf{ACK (Acknowledge)} : Accusé de réception.
    \item \textbf{Handshake TCP} : SYN $\rightarrow$ SYN-ACK $\rightarrow$ ACK.
\end{itemize}

\vspace{0.5cm}

\subsection{Sécurité}

\subsubsection{Question 1 : ISO/IEC 27001}

Norme internationale pour la mise en place d’un \textbf{système de management de la sécurité de l'information (SMSI)}. Elle définit les bonnes pratiques pour protéger la confidentialité, l’intégrité et la disponibilité des données.

\subsubsection{Question 2 : Institutions de cybersécurité}

\begin{itemize}
    \item \textbf{France} :
    \begin{itemize}
        \item ANSSI : Agence nationale de la sécurité des systèmes d'information
        \item CNIL : Autorité de protection des données personnelles
    \end{itemize}
    \item \textbf{International} :
    \begin{itemize}
        \item ENISA : Agence européenne pour la cybersécurité
        \item NIST : National Institute of Standards and Technology (USA)
    \end{itemize}
\end{itemize}

\subsubsection{Question 3 : Moments de vulnérabilité et conséquences}

\subparagraph*{Moments de vulnérabilité}

\begin{itemize}
    \item Connexion à un Wi-Fi public non sécurisé
    \item Téléchargement d’applications non vérifiées
    \item Navigation sur des sites non HTTPS
    \item Utilisation de mots de passe faibles ou réutilisés
    \item Absence de mises à jour des systèmes/appareils
\end{itemize}

\subparagraph*{Conséquences possibles}

\begin{itemize}
    \item Vol de données ou d’identifiants
    \item Piratage de comptes personnels ou professionnels
    \item Usurpation d’identité, chantage, rançongiciel
\end{itemize}

\subparagraph*{Personnes impactées}

\begin{itemize}
    \item \textbf{Famille} : Partage de réseau ou d’appareils
    \item \textbf{Collègues / Clients} : Risques professionnels ou fuite de données
    \item \textbf{Entreprise} : Perte d’image, d’argent ou interruption d’activité
\end{itemize}















\section{Projet Firewall}
\subsection{Configuration du parefeu via iptables}

\subsubsection{Mise en place du serveur web et SSH}

Apr\`es ouverture du terminal, passage en mode super-utilisateur :

\begin{lstlisting}[language=bash]
sudo -s
apt install apache2
ip addr
apt install openssh-server
apt install iptables
apt install tables  # erreur car paquet inexistant
\end{lstlisting}

\subsubsection{R\'einitialisation des r\`egles iptables}

Suppression des r\`egles et cha\^ines pour les tables filter, nat, et mangle. R\'einitialisation des politiques \`a ACCEPT. Affichage des règles du parefeu pour vérifier la bonne application :

\begin{lstlisting}[language=bash]
iptables -nvL
\end{lstlisting}

\subsubsection{Configuration des r\`egles de s\'ecurit\'e}

\`A partir d'une politique de blocage par d\'efaut, j'ai autoris\'e certains ports essentiels (SSH, HTTP, HTTPS) ainsi que le trafic \`a destination de localhost et les connexions \'etablies :

\begin{lstlisting}[language=bash]
iptables -P INPUT DROP
iptables -P FORWARD DROP
iptables -A INPUT -p tcp --dport 22 -m state --state NEW -j ACCEPT
iptables -A INPUT -p tcp --dport 80 -m state --state NEW -j ACCEPT
iptables -A INPUT -p tcp --dport 443 -m state --state NEW -j ACCEPT
iptables -A INPUT -m state --state ESTABLISHED -j ACCEPT
iptables -A INPUT -i lo -j ACCEPT
iptables -A OUTPUT -o lo -j ACCEPT
\end{lstlisting}

\subsection{Supervision avec Logwatch et Rsyslog}

\subsubsection{Installation et configuration}

Installation de Logwatch et de Rsyslog pour assurer la lecture des journaux, Logwatch ne supportant pas directement journalctl :

\begin{lstlisting}[language=bash]
apt install logwatch -y
apt install rsyslog -y
\end{lstlisting}

Ajout dans /etc/rsyslog.d/20-default.conf :

\begin{lstlisting}
auth,authpriv.* /var/log/auth.log
\end{lstlisting}

Cr\'eation du fichier sshd.conf pour Logwatch :

\begin{lstlisting}
nano /etc/logwatch/conf/logfiles/sshd.conf
\end{lstlisting}

Ajout dans /etc/logwatch/conf/logfiles/sshd.conf:

\begin{lstlisting}
LogFile = auth.log
*RemoveHeaders
\end{lstlisting}

\subsubsection{Test}
\begin{lstlisting}[language=bash]
logwatch --range today --service sshd --service http --detail high --format text
\end{lstlisting}
Logwatch repère bien les connections ssh et http.


\subsubsection{Automatisation de la génération de rapports}


Configuration du système de planification de tâches (cron) de manière à enregistrer les activités dans un fichier. (/etc/cron.daily/00logwatch)
Définition d'une variable contenant le chemin du répertoire où les rapports seront sauvegardés :

\begin{lstlisting}[language=bash]
OUTPUT_DIR="/var/log/logwatch"
\end{lstlisting}

Création du dossier s’il n’existait pas déjà, l’option \texttt{-p} évitant les erreurs en cas d’existence préalable :

\begin{lstlisting}[language=bash]
mkdir -p "$OUTPUT_DIR"
\end{lstlisting}

Définition d'une variable contenant le nom du fichier de sortie, comprenant la date du jour, et située dans le répertoire défini :

\begin{lstlisting}[language=bash]
OUTPUT_FILE="$OUTPUT_DIR/logwatch-$(date +%F).log"
\end{lstlisting}

Appel de \texttt{logwatch} avec une analyse de la journée précédente (\texttt{--range yesterday}) pour les services \texttt{sshd} et \texttt{http}, avec une sortie dans le fichier défini :

\begin{lstlisting}[language=bash]
/usr/sbin/logwatch --range yesterday --service sshd --service http --output file --format text --filename "$OUTPUT_FILE"
\end{lstlisting}

Transformation du fichier en fichier exécutable par \texttt{cron} grâce à la commande suivante, où \texttt{+x} donne les droits d’exécution :

\begin{lstlisting}[language=bash]
sudo chmod +x /etc/cron.daily/00logwatch
\end{lstlisting}


\subsection{Protection contre les attaques par force brute avec Fail2ban}

\subsubsection{Installation de fail2ban et vsftpd}

\begin{lstlisting}[language=bash]
apt install fail2ban -y
apt install vsftpd -y
\end{lstlisting}

\subsubsection{Fichier jail.local}

Cr\'eation des jails pour SSH et FTP avec des seuils d'alerte et bannissement personnalis\'es :

\begin{lstlisting}
[sshd]
enabled = true
port = ssh
filter = sshd
logpath = journal
maxretry = 5
findtime = 300
bantime = 3600
backend = systemd

[vsftpd-tbf-ip]
enabled = true
port = ftp
filter = vsftpd
logpath = journal
maxretry = 10
findtime = 300
bantime = 3600
backend = systemd

[vsftpd-tbf-multiuser]
enabled = true
port = ftp
filter = vsftpd-multiuser
logpath = journal
maxretry = 20
findtime = 300
bantime = 3600
backend = systemd
\end{lstlisting}

\subsubsection{Filtres personnalis\'es}

failregex analyse les logs dans journactl qui correspondent à la syntaxe indiquée. Cela permet de définir un filtre qui va trier les adresses IP qui vont être placées dans la jail vsftpd-tbf-ip : 


\textbf{Fichier vsftpd.conf} (/etc/fail2ban/filter.d/vsftpd.conf) :

\begin{lstlisting}
[Definition]
failregex = ^.*pam_unix\(vsftpd:auth\): authentication failure;.*rhost=(::ffff:)?<HOST>
ignoreregex =
\end{lstlisting}

\textbf{Fichier vsftpd-multiuser.conf} :

\begin{lstlisting}
[Definition]
failregex = ^.*pam_unix\(vsftpd:auth\): authentication failure;.*rhost=(::ffff:)?<HOST>
ignoreregex =
\end{lstlisting}

\subsubsection{Red\'emarrage et v\'erifications}

\begin{lstlisting}[language=bash]
sudo systemctl restart fail2ban
sudo systemctl enable fail2ban
\end{lstlisting}

Ajout de la r\`egle pour autoriser le port FTP :

\begin{lstlisting}[language=bash]
iptables -A INPUT -p tcp --dport 21 -m state --state NEW -j ACCEPT
\end{lstlisting}

\textbf{V\'erifications avec Putty (SSH) et FTP depuis Powershell} :

\begin{lstlisting}[language=bash]
fail2ban-client status sshd
fail2ban-client status vsftpd-tbf-ip
\end{lstlisting}


\end{document}
